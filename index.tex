% Options for packages loaded elsewhere
\PassOptionsToPackage{unicode}{hyperref}
\PassOptionsToPackage{hyphens}{url}
%
\documentclass[
  a4paper,
]{scrbook}

\usepackage{amsmath,amssymb}
\usepackage{iftex}
\ifPDFTeX
  \usepackage[T1]{fontenc}
  \usepackage[utf8]{inputenc}
  \usepackage{textcomp} % provide euro and other symbols
\else % if luatex or xetex
  \usepackage{unicode-math}
  \defaultfontfeatures{Scale=MatchLowercase}
  \defaultfontfeatures[\rmfamily]{Ligatures=TeX,Scale=1}
\fi
\usepackage{lmodern}
\ifPDFTeX\else  
    % xetex/luatex font selection
\fi
% Use upquote if available, for straight quotes in verbatim environments
\IfFileExists{upquote.sty}{\usepackage{upquote}}{}
\IfFileExists{microtype.sty}{% use microtype if available
  \usepackage[]{microtype}
  \UseMicrotypeSet[protrusion]{basicmath} % disable protrusion for tt fonts
}{}
\makeatletter
\@ifundefined{KOMAClassName}{% if non-KOMA class
  \IfFileExists{parskip.sty}{%
    \usepackage{parskip}
  }{% else
    \setlength{\parindent}{0pt}
    \setlength{\parskip}{6pt plus 2pt minus 1pt}}
}{% if KOMA class
  \KOMAoptions{parskip=half}}
\makeatother
\usepackage{xcolor}
\setlength{\emergencystretch}{3em} % prevent overfull lines
\setcounter{secnumdepth}{5}
% Make \paragraph and \subparagraph free-standing
\makeatletter
\ifx\paragraph\undefined\else
  \let\oldparagraph\paragraph
  \renewcommand{\paragraph}{
    \@ifstar
      \xxxParagraphStar
      \xxxParagraphNoStar
  }
  \newcommand{\xxxParagraphStar}[1]{\oldparagraph*{#1}\mbox{}}
  \newcommand{\xxxParagraphNoStar}[1]{\oldparagraph{#1}\mbox{}}
\fi
\ifx\subparagraph\undefined\else
  \let\oldsubparagraph\subparagraph
  \renewcommand{\subparagraph}{
    \@ifstar
      \xxxSubParagraphStar
      \xxxSubParagraphNoStar
  }
  \newcommand{\xxxSubParagraphStar}[1]{\oldsubparagraph*{#1}\mbox{}}
  \newcommand{\xxxSubParagraphNoStar}[1]{\oldsubparagraph{#1}\mbox{}}
\fi
\makeatother


\providecommand{\tightlist}{%
  \setlength{\itemsep}{0pt}\setlength{\parskip}{0pt}}\usepackage{longtable,booktabs,array}
\usepackage{calc} % for calculating minipage widths
% Correct order of tables after \paragraph or \subparagraph
\usepackage{etoolbox}
\makeatletter
\patchcmd\longtable{\par}{\if@noskipsec\mbox{}\fi\par}{}{}
\makeatother
% Allow footnotes in longtable head/foot
\IfFileExists{footnotehyper.sty}{\usepackage{footnotehyper}}{\usepackage{footnote}}
\makesavenoteenv{longtable}
\usepackage{graphicx}
\makeatletter
\def\maxwidth{\ifdim\Gin@nat@width>\linewidth\linewidth\else\Gin@nat@width\fi}
\def\maxheight{\ifdim\Gin@nat@height>\textheight\textheight\else\Gin@nat@height\fi}
\makeatother
% Scale images if necessary, so that they will not overflow the page
% margins by default, and it is still possible to overwrite the defaults
% using explicit options in \includegraphics[width, height, ...]{}
\setkeys{Gin}{width=\maxwidth,height=\maxheight,keepaspectratio}
% Set default figure placement to htbp
\makeatletter
\def\fps@figure{htbp}
\makeatother

\makeatletter
\@ifpackageloaded{tcolorbox}{}{\usepackage[skins,breakable]{tcolorbox}}
\@ifpackageloaded{fontawesome5}{}{\usepackage{fontawesome5}}
\definecolor{quarto-callout-color}{HTML}{909090}
\definecolor{quarto-callout-note-color}{HTML}{0758E5}
\definecolor{quarto-callout-important-color}{HTML}{CC1914}
\definecolor{quarto-callout-warning-color}{HTML}{EB9113}
\definecolor{quarto-callout-tip-color}{HTML}{00A047}
\definecolor{quarto-callout-caution-color}{HTML}{FC5300}
\definecolor{quarto-callout-color-frame}{HTML}{acacac}
\definecolor{quarto-callout-note-color-frame}{HTML}{4582ec}
\definecolor{quarto-callout-important-color-frame}{HTML}{d9534f}
\definecolor{quarto-callout-warning-color-frame}{HTML}{f0ad4e}
\definecolor{quarto-callout-tip-color-frame}{HTML}{02b875}
\definecolor{quarto-callout-caution-color-frame}{HTML}{fd7e14}
\makeatother
\makeatletter
\@ifpackageloaded{caption}{}{\usepackage{caption}}
\AtBeginDocument{%
\ifdefined\contentsname
  \renewcommand*\contentsname{Table of contents}
\else
  \newcommand\contentsname{Table of contents}
\fi
\ifdefined\listfigurename
  \renewcommand*\listfigurename{List of Figures}
\else
  \newcommand\listfigurename{List of Figures}
\fi
\ifdefined\listtablename
  \renewcommand*\listtablename{List of Tables}
\else
  \newcommand\listtablename{List of Tables}
\fi
\ifdefined\figurename
  \renewcommand*\figurename{Figure}
\else
  \newcommand\figurename{Figure}
\fi
\ifdefined\tablename
  \renewcommand*\tablename{Table}
\else
  \newcommand\tablename{Table}
\fi
}
\@ifpackageloaded{float}{}{\usepackage{float}}
\floatstyle{ruled}
\@ifundefined{c@chapter}{\newfloat{codelisting}{h}{lop}}{\newfloat{codelisting}{h}{lop}[chapter]}
\floatname{codelisting}{Listing}
\newcommand*\listoflistings{\listof{codelisting}{List of Listings}}
\makeatother
\makeatletter
\makeatother
\makeatletter
\@ifpackageloaded{caption}{}{\usepackage{caption}}
\@ifpackageloaded{subcaption}{}{\usepackage{subcaption}}
\makeatother

\ifLuaTeX
  \usepackage{selnolig}  % disable illegal ligatures
\fi
\usepackage{bookmark}

\IfFileExists{xurl.sty}{\usepackage{xurl}}{} % add URL line breaks if available
\urlstyle{same} % disable monospaced font for URLs
\hypersetup{
  pdftitle={TQM Lectures 2024},
  hidelinks,
  pdfcreator={LaTeX via pandoc}}


\title{TQM Lectures 2024}
\author{}
\date{}

\begin{document}
\frontmatter
\maketitle


\mainmatter
These are the materials for Theories of Quantum Matter, a Part III
Physics course at the University of Cambridge taught in Michaelmas 2024.
The materials for previous versions of the course can be found
\href{https://austen.uk/courses/}{here}. While these are largely
unchanged, you should regard this site as the authoritative version for
2024.

The source of these notes is
\href{https://github.com/AustenLamacraft/tqm-quarto}{this GitHub repo}.
If you find a typo, other problem or have any other comment on the
material, please
\href{https://github.com/AustenLamacraft/tqm-quarto/issues}{raise an
issue} there.

\href{https://github.com/AustenLamacraft/tqm-quarto/discussions/}{The
discussions section} is a good place to ask each other questions. Try
it!

If you ever want access to the \(\LaTeX\) for the mathematical
expressions, right click and select \texttt{Show\ Math\ As}
\textgreater{} \texttt{TeX\ Commands}.

\begin{tcolorbox}[enhanced jigsaw, breakable, opacitybacktitle=0.6, coltitle=black, bottomrule=.15mm, title=\textcolor{quarto-callout-tip-color}{\faLightbulb}\hspace{0.5em}{Check}, opacityback=0, toprule=.15mm, leftrule=.75mm, left=2mm, titlerule=0mm, rightrule=.15mm, colframe=quarto-callout-tip-color-frame, toptitle=1mm, colback=white, bottomtitle=1mm, arc=.35mm, colbacktitle=quarto-callout-tip-color!10!white]

These ``Check'' boxes provide small exercises to check your
understanding as you read the notes. Make sure you do them!

\end{tcolorbox}

\begin{tcolorbox}[enhanced jigsaw, breakable, opacitybacktitle=0.6, coltitle=black, bottomrule=.15mm, title=\textcolor{quarto-callout-note-color}{\faInfo}\hspace{0.5em}{Note}, opacityback=0, toprule=.15mm, leftrule=.75mm, left=2mm, titlerule=0mm, rightrule=.15mm, colframe=quarto-callout-note-color-frame, toptitle=1mm, colback=white, bottomtitle=1mm, arc=.35mm, colbacktitle=quarto-callout-note-color!10!white]

``Note'' boxes provide extra context that you may find interesting but
aren't essential. Except this one. This one's important.

\end{tcolorbox}

Finally\ldots{}

\begin{tcolorbox}[enhanced jigsaw, breakable, opacitybacktitle=0.6, coltitle=black, bottomrule=.15mm, title=\textcolor{quarto-callout-warning-color}{\faExclamationTriangle}\hspace{0.5em}{Warning}, opacityback=0, toprule=.15mm, leftrule=.75mm, left=2mm, titlerule=0mm, rightrule=.15mm, colframe=quarto-callout-warning-color-frame, toptitle=1mm, colback=white, bottomtitle=1mm, arc=.35mm, colbacktitle=quarto-callout-warning-color!10!white]

In this course \(\hbar = 1\)!

\end{tcolorbox}

\chapter*{Lectures}\label{lectures}
\addcontentsline{toc}{chapter}{Lectures}

Take place in the Small Lecture Theatre at the Cavendish Laboratory from
09.30-11.00 on Tuesdays and Thursdays.

\chapter*{Supervisions}\label{supervisions}
\addcontentsline{toc}{chapter}{Supervisions}

There will be four supervisions, taking place in the weeks commencing
22/10, 5/11, 19/11, and 3/12.


\backmatter


\end{document}
