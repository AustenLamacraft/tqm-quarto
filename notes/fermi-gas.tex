% Options for packages loaded elsewhere
% Options for packages loaded elsewhere
\PassOptionsToPackage{unicode}{hyperref}
\PassOptionsToPackage{hyphens}{url}
%
\documentclass[
  a4paper,
]{scrbook}
\usepackage{xcolor}
\usepackage{amsmath,amssymb}
\setcounter{secnumdepth}{5}
\usepackage{iftex}
\ifPDFTeX
  \usepackage[T1]{fontenc}
  \usepackage[utf8]{inputenc}
  \usepackage{textcomp} % provide euro and other symbols
\else % if luatex or xetex
  \usepackage{unicode-math} % this also loads fontspec
  \defaultfontfeatures{Scale=MatchLowercase}
  \defaultfontfeatures[\rmfamily]{Ligatures=TeX,Scale=1}
\fi
\usepackage{lmodern}
\ifPDFTeX\else
  % xetex/luatex font selection
\fi
% Use upquote if available, for straight quotes in verbatim environments
\IfFileExists{upquote.sty}{\usepackage{upquote}}{}
\IfFileExists{microtype.sty}{% use microtype if available
  \usepackage[]{microtype}
  \UseMicrotypeSet[protrusion]{basicmath} % disable protrusion for tt fonts
}{}
\makeatletter
\@ifundefined{KOMAClassName}{% if non-KOMA class
  \IfFileExists{parskip.sty}{%
    \usepackage{parskip}
  }{% else
    \setlength{\parindent}{0pt}
    \setlength{\parskip}{6pt plus 2pt minus 1pt}}
}{% if KOMA class
  \KOMAoptions{parskip=half}}
\makeatother
% Make \paragraph and \subparagraph free-standing
\makeatletter
\ifx\paragraph\undefined\else
  \let\oldparagraph\paragraph
  \renewcommand{\paragraph}{
    \@ifstar
      \xxxParagraphStar
      \xxxParagraphNoStar
  }
  \newcommand{\xxxParagraphStar}[1]{\oldparagraph*{#1}\mbox{}}
  \newcommand{\xxxParagraphNoStar}[1]{\oldparagraph{#1}\mbox{}}
\fi
\ifx\subparagraph\undefined\else
  \let\oldsubparagraph\subparagraph
  \renewcommand{\subparagraph}{
    \@ifstar
      \xxxSubParagraphStar
      \xxxSubParagraphNoStar
  }
  \newcommand{\xxxSubParagraphStar}[1]{\oldsubparagraph*{#1}\mbox{}}
  \newcommand{\xxxSubParagraphNoStar}[1]{\oldsubparagraph{#1}\mbox{}}
\fi
\makeatother


\usepackage{longtable,booktabs,array}
\usepackage{calc} % for calculating minipage widths
% Correct order of tables after \paragraph or \subparagraph
\usepackage{etoolbox}
\makeatletter
\patchcmd\longtable{\par}{\if@noskipsec\mbox{}\fi\par}{}{}
\makeatother
% Allow footnotes in longtable head/foot
\IfFileExists{footnotehyper.sty}{\usepackage{footnotehyper}}{\usepackage{footnote}}
\makesavenoteenv{longtable}
\usepackage{graphicx}
\makeatletter
\newsavebox\pandoc@box
\newcommand*\pandocbounded[1]{% scales image to fit in text height/width
  \sbox\pandoc@box{#1}%
  \Gscale@div\@tempa{\textheight}{\dimexpr\ht\pandoc@box+\dp\pandoc@box\relax}%
  \Gscale@div\@tempb{\linewidth}{\wd\pandoc@box}%
  \ifdim\@tempb\p@<\@tempa\p@\let\@tempa\@tempb\fi% select the smaller of both
  \ifdim\@tempa\p@<\p@\scalebox{\@tempa}{\usebox\pandoc@box}%
  \else\usebox{\pandoc@box}%
  \fi%
}
% Set default figure placement to htbp
\def\fps@figure{htbp}
\makeatother


% definitions for citeproc citations
\NewDocumentCommand\citeproctext{}{}
\NewDocumentCommand\citeproc{mm}{%
  \begingroup\def\citeproctext{#2}\cite{#1}\endgroup}
\makeatletter
 % allow citations to break across lines
 \let\@cite@ofmt\@firstofone
 % avoid brackets around text for \cite:
 \def\@biblabel#1{}
 \def\@cite#1#2{{#1\if@tempswa , #2\fi}}
\makeatother
\newlength{\cslhangindent}
\setlength{\cslhangindent}{1.5em}
\newlength{\csllabelwidth}
\setlength{\csllabelwidth}{3em}
\newenvironment{CSLReferences}[2] % #1 hanging-indent, #2 entry-spacing
 {\begin{list}{}{%
  \setlength{\itemindent}{0pt}
  \setlength{\leftmargin}{0pt}
  \setlength{\parsep}{0pt}
  % turn on hanging indent if param 1 is 1
  \ifodd #1
   \setlength{\leftmargin}{\cslhangindent}
   \setlength{\itemindent}{-1\cslhangindent}
  \fi
  % set entry spacing
  \setlength{\itemsep}{#2\baselineskip}}}
 {\end{list}}
\usepackage{calc}
\newcommand{\CSLBlock}[1]{\hfill\break\parbox[t]{\linewidth}{\strut\ignorespaces#1\strut}}
\newcommand{\CSLLeftMargin}[1]{\parbox[t]{\csllabelwidth}{\strut#1\strut}}
\newcommand{\CSLRightInline}[1]{\parbox[t]{\linewidth - \csllabelwidth}{\strut#1\strut}}
\newcommand{\CSLIndent}[1]{\hspace{\cslhangindent}#1}



\setlength{\emergencystretch}{3em} % prevent overfull lines

\providecommand{\tightlist}{%
  \setlength{\itemsep}{0pt}\setlength{\parskip}{0pt}}



 


\makeatletter
\@ifpackageloaded{tcolorbox}{}{\usepackage[skins,breakable]{tcolorbox}}
\@ifpackageloaded{fontawesome5}{}{\usepackage{fontawesome5}}
\definecolor{quarto-callout-color}{HTML}{909090}
\definecolor{quarto-callout-note-color}{HTML}{0758E5}
\definecolor{quarto-callout-important-color}{HTML}{CC1914}
\definecolor{quarto-callout-warning-color}{HTML}{EB9113}
\definecolor{quarto-callout-tip-color}{HTML}{00A047}
\definecolor{quarto-callout-caution-color}{HTML}{FC5300}
\definecolor{quarto-callout-color-frame}{HTML}{acacac}
\definecolor{quarto-callout-note-color-frame}{HTML}{4582ec}
\definecolor{quarto-callout-important-color-frame}{HTML}{d9534f}
\definecolor{quarto-callout-warning-color-frame}{HTML}{f0ad4e}
\definecolor{quarto-callout-tip-color-frame}{HTML}{02b875}
\definecolor{quarto-callout-caution-color-frame}{HTML}{fd7e14}
\makeatother
\makeatletter
\@ifpackageloaded{caption}{}{\usepackage{caption}}
\AtBeginDocument{%
\ifdefined\contentsname
  \renewcommand*\contentsname{Table of contents}
\else
  \newcommand\contentsname{Table of contents}
\fi
\ifdefined\listfigurename
  \renewcommand*\listfigurename{List of Figures}
\else
  \newcommand\listfigurename{List of Figures}
\fi
\ifdefined\listtablename
  \renewcommand*\listtablename{List of Tables}
\else
  \newcommand\listtablename{List of Tables}
\fi
\ifdefined\figurename
  \renewcommand*\figurename{Figure}
\else
  \newcommand\figurename{Figure}
\fi
\ifdefined\tablename
  \renewcommand*\tablename{Table}
\else
  \newcommand\tablename{Table}
\fi
}
\@ifpackageloaded{float}{}{\usepackage{float}}
\floatstyle{ruled}
\@ifundefined{c@chapter}{\newfloat{codelisting}{h}{lop}}{\newfloat{codelisting}{h}{lop}[chapter]}
\floatname{codelisting}{Listing}
\newcommand*\listoflistings{\listof{codelisting}{List of Listings}}
\makeatother
\makeatletter
\makeatother
\makeatletter
\@ifpackageloaded{caption}{}{\usepackage{caption}}
\@ifpackageloaded{subcaption}{}{\usepackage{subcaption}}
\makeatother
\usepackage{bookmark}
\IfFileExists{xurl.sty}{\usepackage{xurl}}{} % add URL line breaks if available
\urlstyle{same}
\hypersetup{
  pdftitle={Fermi Gas},
  hidelinks,
  pdfcreator={LaTeX via pandoc}}


\title{Fermi Gas}
\author{}
\date{}
\begin{document}
\frontmatter
\maketitle


\mainmatter
\providecommand{\cE}{\mathcal{E}}
\providecommand{\cH}{\mathcal{H}}
\providecommand{\cN}{\mathcal{N}}
\providecommand{\cO}{\mathcal{O}}
\providecommand{\cP}{\mathcal{P}}
\providecommand{\br}{\mathbf{r}}
\providecommand{\bp}{\mathbf{p}}
\providecommand{\bk}{\mathbf{k}}
\providecommand{\bq}{\mathbf{q}}
\providecommand{\bv}{\mathbf{v}}
\providecommand{\brN}{\br_1, \ldots, \br_N}
\providecommand{\xN}{x_1, \ldots, x_N}
\providecommand{\zN}{z_1, \ldots, z_N}
\providecommand{\pop}{\psi^{\vphantom{\dagger}}}
\providecommand{\pdop}{\psi^\dagger}
\providecommand{\Pop}{\Psi^{\vphantom{\dagger}}}
\providecommand{\Pdop}{\Psi^\dagger}
\providecommand{\Phop}{\Phi^{\vphantom{\dagger}}}
\providecommand{\Phdop}{\Phi^\dagger}
\providecommand{\phop}{\phi^{\vphantom{\dagger}}}
\providecommand{\phdop}{\phi^\dagger}
\providecommand{\aop}{a^{\vphantom{\dagger}}}
\providecommand{\adop}{a^\dagger}
\providecommand{\bop}{b^{\vphantom{\dagger}}}
\providecommand{\bdop}{b^\dagger}
\providecommand{\cop}{c^{\vphantom{\dagger}}}
\providecommand{\cdop}{c^\dagger}
\providecommand{\alop}{\alpha^{\vphantom{\dagger}}}
\providecommand{\aldop}{\alpha^\dagger}
\providecommand{\bra}[1]{\langle{#1}\rvert}
\providecommand{\ket}[1]{\lvert{#1}\rangle}
\providecommand{\inner}[2]{\langle{#1}\rvert #2 \rangle}
\providecommand{\braket}[3]{\langle{#1}\rvert #2 \lvert #3 \rangle}
\providecommand{\sgn}{\mathrm{sgn}}
\providecommand{\abs}[1]{\lvert{#1}\rvert}
\providecommand{\Nop}{\mathsf{N}^{\vphantom{\dagger}}}

A Fermi gas with weak interactions provides an example of one of the
`standard models' of condensed matter physics: Landau's Fermi liquid
theory.

Reading: Landau et al. (1980)

\begin{center}\rule{0.5\linewidth}{0.5pt}\end{center}

\chapter{Weakly Interacting Fermi
Gas}\label{weakly-interacting-fermi-gas}

We are going to study the following simple model of a Fermi gas with
short-ranged interactions

\[
H = \int d\br\left[ \sum_{s=\uparrow,\downarrow}\frac{1}{2m}\nabla\pdop_s\cdot\nabla\pop_s + U_0 \pdop_\uparrow\pdop_\downarrow\pop_\downarrow\pop_\uparrow\right].
\]

As with the Bose gas, it's most convenient to work in momentum space

\[
H =\sum_{\bk,s} \epsilon(\bk)\adop_{\bk,s}\aop_{\bk,s} + \overbrace{\frac{U_0}{V}\sum_{\bk_1+\bk_2=\bk_3+\bk_4} \adop_{\bk_1,\uparrow}\adop_{\bk_2,\downarrow}\aop_{\bk_3,\downarrow}\aop_{\bk_4,\uparrow}}^{\equiv H_\text{int}},
\]

with \(\epsilon(\bk)=\bk^2/2m\), and \(V\) the volume. At \(U_0=0\) the
eigenstates are of course the product states of single particle momentum
states specified by the occupancies \(N_{s}(\bk) = 0,1\). The ground
state is the Fermi sphere of radius \(k_\text{F}\) in momentum space
with \(N_{s}(\bk) = \theta(k_F-\abs{\bk})\). Low energy excited states
will have \(N_{s}(\bk)=1\) for \(\abs{\bk}\ll k_\text{F}\) and
\(N_{s}(\bk)=0\) for \(\abs{\bk}\gg k_\text{F}\). In perturbation theory
we may still \emph{label} the eigenstates by these occupation numbers
even though the eigenstates are no longer occupation number eigenstates.
Instead, we say that these labels give the occupation numbers of
\textbf{quasiparticles} with fermionic statistics. The energy of the
eigenstates can then be expressed in terms of the quasiparticle
distribution.

Without interactions the energy of a state \(\ket{\mathbf{N}}\) is

\[
E^{(0)}(\mathbf{N}) = \sum_{\bk,s} \epsilon(\bk)N_{s}(\bk).
\]

In the presence of interactions this function is no longer linear in the
occupation numbers. The second order expansion of the energy in terms of
the deviation of the occupancies from the ground state values is the key
ingredient of Landau's theory.

\chapter{Perturbation Theory to Second
Order}\label{perturbation-theory-to-second-order}

The standard expressions for the perturbed energy to second order in the
perturbation are of course

\begin{equation}\phantomsection\label{eq-2nd}{
\begin{aligned}
E^{(1)}(\mathbf{N}) &= \braket{\mathbf{N}}{H_\text{int}}{\mathbf{N}}\\
E^{(2)}(\mathbf{N}) &= \sum_{\mathbf{N}'\neq \mathbf N}\frac{\abs{\braket{\mathbf{N'}}{H_\text{int}}{\mathbf{N}}}^2}{E^{(0)}(\mathbf{N})-E^{(0)}(\mathbf{N}')}.
\end{aligned}
}\end{equation}

The first order correction is easily found to be

\[
E^{(1)}(\mathbf{N}) = \frac{U_0}{V} \sum_{\bk,\bk'} N_{\uparrow}(\bk)N_{\downarrow}(\bk') = \frac{U_0}{V}N_\uparrow N_\downarrow.
\]

This is just the energy we found when we discussed the Stoner criterion
in \href{../notes/more-second-quantization.qmd}{Lecture 6}. For the
second order correction we need the matrix element
\(\braket{\mathbf{N}'}{H_\text{int}}{\mathbf{N}}\), which is nonzero if

\[
\begin{aligned}
N'_{\bk_1,\uparrow} = N_{\uparrow}(\bk_1) + 1, \quad N'_{\downarrow}(\bk_2) = N_{\downarrow}(\bk_2) + 1\\
N'_{\downarrow}(\bk_3) = N_{\downarrow}(\bk_3) - 1, \quad N'_{\uparrow}(\bk_4) = N_{\uparrow}(\bk_4) - 1,
\end{aligned}
\]

for \(\bk_i\) satisfying \(\bk_1+\bk_2=\bk_3+\bk_4\). In this case

\[
\braket{\mathbf{N}'}{H_\text{int}}{\mathbf{N}} = \frac{U_0}{V} \left(1-N_{\uparrow}(\bk_1)\right)\left(1-N_{\downarrow}(\bk_2)\right)N_{\downarrow}(\bk_3)N_{\uparrow}(\bk_4),
\]

(ignoring any coinciding momenta) where the occupancies are either zero
or one. In this way we end up with the second order correction

\begin{equation}\phantomsection\label{eq-E2}{
E^{(2)}(\mathbf{N}) = \left(\frac{U_0}{V}\right)^2 \sum_{\bk_1+\bk_2=\bk_3+\bk_4}\frac{\left(1-N_{\uparrow}(\bk_1)\right)\left(1-N_{\downarrow}(\bk_2)\right)N_{\downarrow}(\bk_3)N_{\uparrow}(\bk_4)}{\epsilon(\bk_3)+\epsilon(\bk_4)-\epsilon(\bk_1)-\epsilon(\bk_2)}.
}\end{equation}

\section{\texorpdfstring{Landau \(f\)
function}{Landau f function}}\label{landau-f-function}

Evaluating Equation~\ref{eq-E2}, even for the ground state, is a bit
arduous on account of the three independent momentum sums. Fortunately,
we are more interested in how the \emph{excitation energies} are
affected by interactions. This means we focus on the \emph{change} in
the occupation relative to the ground state, denoted by \(n_s(\bk)\) and
defined through

\[
N_{s}(\bk) = \theta(k_F-\abs{\bk}) + n_{s}(\bk)
\]

This might seem a bit odd given that \(N_{s}(\bk)=\pm 1\). You should
think of this expansion in terms of the continuum limit, where the
\(\bk\) values become finely spaced. In this limit \(n_\bk\) represents
the \emph{mean} deviation from the Fermi sphere in that region of
\(\bk\)-space.

\begin{figure}[H]

{\centering \includegraphics[width=0.7\linewidth,height=\textheight,keepaspectratio]{../assets/FermiN.png}

}

\caption{\(n_\bk\) as a smoothed deviation from the Fermi step.}

\end{figure}%

The excitation energy above the ground state of an eigenstate labelled
by occupancies \(N_\bk\) can then be expanded in \(n_\bk\)

\begin{equation}\phantomsection\label{eq-fdef}{
\Delta E = \sum_{\bk,s} \varepsilon_s(\bk)n_{s}(\bk) + \frac{1}{2V}\sum_{\bk, s,\bk', s'} f_{s^{}s'}(\bk,\bk')n_{s}(\bk)n_{s'}(\bk').
}\end{equation}

Note that \(f_{s^{}s'}(\bk,\bk')\) is symmetric under exchange of
\(\bk\) and \(\bk'\), and \(s\) and \(s'\). This expansion is the key
idea in the theory of the Fermi liquid. Although we will calculate the
quasiparticle energy \(\varepsilon_s(\bk)\) and interaction function
\(f_{s^{}s'}(\bk,\bk')\) using perturbation theory, Landau's idea was
that \emph{any} interacting Fermi system could be described in similar
terms, as long as the ground state does not change abruptly as we
increase the interaction from zero (normally a thought experiment!). An
example of an abrupt change would be a transition from liquid to solid
(crystallization).

To first order in the interaction we have the not-so-interesting result

\[
\begin{aligned}
\varepsilon_s(\bk) &= \epsilon(\bk) + \frac{U_0 N_{\bar s}}{V}+\cdots\\
f_{\uparrow\downarrow} &= f_{\downarrow\uparrow} = U_0+\cdots,\quad f_{\uparrow\uparrow}=f_{\downarrow\downarrow}=0+\cdots,
\label{fermi_first}
\end{aligned}
\]

where the meaning of the \(\bar s\) is \(\bar\uparrow=\downarrow\),
\(\bar\downarrow=\uparrow\). The second order contributions to the
\(f\)-function are more interesting, however. For example

\[
\begin{aligned}
f_{\uparrow\uparrow}(\bk,\bk') = -\frac{U_0^2}{V}\left[\sum_{\bk+\bk_3=\bk'+\bk_2} \frac{N_{\downarrow}(\bk_3)(1-N_{\downarrow}(\bk_2))}{\epsilon(\bk)+\epsilon(\bk_3)-\epsilon(\bk')-\epsilon(\bk_2)}\right.\nonumber\\
\left.+\sum_{\bk'+\bk_3=\bk+\bk_2}\frac{N_{\downarrow}(\bk_3)(1-N_{\downarrow}(\bk_2))}{\epsilon(\bk')+\epsilon(\bk_3)-\epsilon(\bk)-\epsilon(\bk_2)}\right].
\end{aligned}
\]

We will be interested in the low temperature limit, in which
\(n_{s}(\bk)\) is non-zero only in a very narrow region of size
\(k_\text{B} T\) around the Fermi surface. In this limit we can take the
\(\abs{\bk}=\abs{\bk'}=k_\text{F}\), so that

\[
\begin{aligned}
f_{\uparrow\uparrow}(\bk,\bk') = -\frac{U_0^2}{V}\left[\sum_{\bk+\bk_3=\bk'+\bk_2} \frac{N(\bk_3)(1-N(\bk_2))}{\epsilon(\bk_3)-\epsilon(\bk_2)}
+\sum_{\bk'+\bk_3=\bk+\bk_2}\frac{N(\bk_3)(1-N(\bk_2))}{\epsilon(\bk_3)-\epsilon(\bk_2)}\right].
\end{aligned}
\]

In which we have assumed the state around which we expand is
unpolarized, i.e.~\(N_{s}(\bk)\) is independent of \(s\). In this case
\(f_{\uparrow\uparrow}(\bk,\bk')=f_{\uparrow\uparrow}(\bk,\bk')\)

The expression for
\(f_{\uparrow\downarrow}(\bk,\bk')=f_{\downarrow\uparrow}(\bk,\bk')\) is
more complicated

\[
\begin{aligned}
f_{\uparrow\downarrow}(\bk,\bk') = U_0 + f_{\uparrow\uparrow}(\bk,\bk') +\frac{U_0^2}{V}\left[\sum_{\bk+\bk'=\bk_3+\bk_4}\frac{N(\bk_3)N(\bk_4)}{\epsilon(\bk_3)+\epsilon(\bk_4)-2E_\text{F}}\right.\nonumber\\
\left.\sum_{\bk+\bk'=\bk_1+\bk_2}\frac{(1-N(\bk_1))(1-N(\bk_2))}{2E_\text{F}-\epsilon(\bk_1)-\epsilon(\bk_2)}\right].
\end{aligned}
\]

Evaluating these expressions is simpler than calculating
Equation~\ref{eq-E2}, as we have only one independent momentum. The
\emph{new feature} that comes at second order is a nontrivial dependence
of \(f_{s^{}s'}(\bk,\bk')\) on the angle between \(\bk\) and \(\bk'\).

It's a bit fiddly to get at, but let's work it out for the simpler case
of \(f_{\uparrow\uparrow}(\bk,\bk')\)!. The continuum limit is

\begin{equation}\phantomsection\label{eq-fcont}{
\begin{aligned}
f_{\uparrow\uparrow}(\bk,\bk') = \frac{U_0^2}{(2\pi)^3}\left[\int_{\substack{\abs{\bk_3}<k_\text{F},\abs{\bk_2}>k_\text{F}\\ \bk+\bk_3=\bk'+\bk_2 }} \frac{d\bk_3}{\epsilon(\bk_2)-\epsilon(\bk_3)}\right.\nonumber\\
\left.+\int_{\substack{\abs{\bk_3}<k_\text{F},\abs{\bk_2}>k_\text{F}\\ \bk'+\bk_3=\bk+\bk_2 }}\frac{d\bk_3}{\epsilon(\bk_2)-\epsilon(\bk_3)}\right].
\end{aligned}
}\end{equation}

So we need to find the integral

\[
\int_{\substack{\abs{\bk_3}<k_\text{F},\abs{\bk_2}>k_\text{F}\\ \bk+\bk_3=\bk'+\bk_2 }} \frac{d\bk_3}{\epsilon(\bk_2)-\epsilon(\bk_3)}.
\]

Note that the denominator can be written

\[
\epsilon(\bk_2)-\epsilon(\bk_3)= \frac{1}{2m}\left(\bk_2+\bk_3\right)\cdot\left(\bk_2-\bk_3\right) = \frac{1}{2m}\left(\bk_2+\bk_3\right)\cdot\left(\bk-\bk'\right),
\]

where \(\bk-\bk'\) is fixed. Writing

\[
\mathbf{K} = \frac{1}{2}\left(\bk_2+\bk_3\right),\quad \bq = \frac{1}{2}\left(\bk_2-\bk_3\right),
\]

\begin{figure}[H]

{\centering \includegraphics[width=0.6\linewidth,height=\textheight,keepaspectratio]{../assets/FermiGeometry.png}

}

\caption{Geometry of the integral for
\(f_{\uparrow\uparrow}(\bk,\bk')\).}

\end{figure}%

the denominator becomes

\[
\epsilon(\bk_2)-\epsilon(\bk_3) = \frac{2}{m}\mathbf{K}\cdot\bq,
\]

for fixed \(\bq\). Thus, only the angle \(\theta\) between
\(\mathbf{K}\) and \(\bq\) enters the integral. The conditions
\(\abs{\bk_2}>k_\text{F}\) and \(\abs{\bk_3}<k_\text{F}\) become

\[
\begin{align}
\left(\mathbf{K}+\bq\right)^2>k_\text{F}^2,\quad \left(\mathbf{K}-\bq\right)^2<k_\text{F}^2,
\end{align}
\]

which gives the range of \(K_-(\theta)<\abs{\mathbf{K}}<K_+(\theta)\)

\[
K_{\pm}(\theta)=\pm q\abs{\cos\theta}+\sqrt{k_\text{F}^2-q^2\sin^2\theta},
\]

and we must have \(\theta<\pi/2\). In terms of these variables the
integral becomes

\[
\begin{align}
\int_{\substack{\abs{\bk_3}<k_\text{F},\abs{\bk_2}>k_\text{F}\\ \bk+\bk_3=\bk'+\bk_2 }} \frac{d\bk_3}{\epsilon(\bk_2)-\epsilon(\bk_3)}&=
\pi m\int_0^{\pi/2} d\theta \int_{K_-(\theta)}^{K_+(\theta)} \frac{K\sin\theta}{q\cos\theta} dK\nonumber\\
&=2\pi m\int_0^{\pi/2} d\theta \sin\theta \sqrt{k_\text{F}^2-q^2\sin^2\theta}.
\end{align}
\]

The other integral in Equation~\ref{eq-fcont} is the same but in the
interval \((\pi/2,\pi)\). Thus we have Finally

\[
\begin{align}
f_{\uparrow\uparrow}(\bk,\bk') &= \frac{U_0^2 m}{(2\pi)^2} \int_0^{\pi} d\theta \sin\theta \sqrt{k_\text{F}^2-q^2\sin^2\theta}\nonumber\\
&=\frac{U_0^2 m k_\text{F}}{(2\pi)^2}\left[1 - \frac{\cos^2\phi/2}{2\sin\phi/2}\log\left(\frac{1-\sin\phi/2}{1+\sin\phi/2}\right)\right].
\end{align}
\]

Here \(\phi\) is the angle between \(\bk\) and \(\bk'\)
i.e.~\(\abs{\bk-\bk'}=2q=2k_\text{F}\sin\phi/2\). We won't go through
the calculation of \(f_{\uparrow\downarrow}(\bk,\bk')\), but just record
the final answer.

Before doing that, let us first write the quadratic contribution to
Equation~\ref{eq-fdef} in an alternative way that allows us to
generalize to a fully rotationally invariant formulation (see the
\hyperref[rotationally-invariant-formulation]{Appendix})

\begin{equation}\phantomsection\label{eq-f-rewrite}{
\begin{align}
\frac{1}{2V \nu(E_F)}\sum_{\bk,\bk'} &\left(F(\phi)\left[n_\uparrow(\bk)+n_\downarrow(\bk)\right]\left[n_\uparrow(\bk')+n_\downarrow(\bk')\right]\right.\\ &\left.+ G(\phi)\left[n_\uparrow(\bk)-n_\downarrow(\bk)\right]\left[n_\uparrow(\bk')-n_\downarrow(\bk')\right]\right).
\end{align}
}\end{equation}

Here we have defined

\begin{equation}\phantomsection\label{eq-fgdef}{
\begin{align}
F(\phi)= \frac{\nu(E_\text{F})}{2}\left[f_{\uparrow\uparrow}(\bk,\bk') + f_{\uparrow\downarrow}(\bk,\bk')\right]\\
G(\phi)= \frac{\nu(E_\text{F})}{2}\left[f_{\uparrow\uparrow}(\bk,\bk') - f_{\uparrow\downarrow}(\bk,\bk')\right].
\end{align}
}\end{equation}

\(F(\phi)\) and \(G(\phi)\) have been made dimensionless by scaling by
the density of states per unit volume at the Fermi surface
\(\nu(E_\text{F})\equiv k_{\text{F}}m/\pi^2\).

The explicit form of the functions \(F(\phi)\) and \(G(\phi)\) is (see
Landau et al. (1980))

\begin{equation}\phantomsection\label{eq-ffinal}{
\begin{align}
F(\phi)=\frac{\nu(E_\text{F})U_0}{2}\left[\left(1+ \frac{\nu(E_\text{F})U_0 }{2}\left[2+\frac{\cos\phi}{2\sin\phi/2}\log\frac{1+\sin\phi/2}{1-\sin\phi/2}\right]\right)\right]\\
G(\phi)=\frac{\nu(E_\text{F})U_0}{2}\left[\left(1+ \frac{\nu(E_\text{F})U_0}{2}\left[1-\frac{1}{2}\sin\phi/2\log\frac{1+\sin\phi/2}{1-\sin\phi/2}\right]\right)\right].
\end{align} 
}\end{equation}

\begin{tcolorbox}[enhanced jigsaw, colframe=quarto-callout-warning-color-frame, leftrule=.75mm, arc=.35mm, rightrule=.15mm, bottomrule=.15mm, left=2mm, toprule=.15mm, breakable, opacityback=0, colback=white]

The message to take away from this calculation is not the detailed
functional form of Equation~\ref{eq-ffinal}, but the fact that the
interaction between quasiparticles in an interacting Fermi gas is
defined in terms of a pair of functions \(F(\phi)\) and \(G(\phi)\)

\end{tcolorbox}

\section{\texorpdfstring{Quasiparticle energy
\(\varepsilon_s(\bk)\)}{Quasiparticle energy \textbackslash varepsilon\_s(\textbackslash bk)}}\label{quasiparticle-energy-varepsilon_sbk}

So far we haven't had much to say about the quasiparticle energy
\(\varepsilon_s(\bk)\) introduced in Equation~\ref{eq-fdef}. Evaluating
the second order correction is going to be difficult, as it will involve
two momentum integrations instead of one. What can we say on general
grounds? We expect that

\begin{equation}\phantomsection\label{eq-vfdef}{
\varepsilon_s(\bk) - E_\text{F} = v_\text{F}(\abs{\bk}-k_\text{F}).
}\end{equation}

The Fermi velocity \(v_\text{F}\) defined by this expression may be
altered by the interactions, allowing us to define an \textbf{effective
mass}

\begin{equation}\phantomsection\label{eq-mdef}{
m_* = \frac{k_\text{F}}{v_\text{F}}.
}\end{equation}

Fortunately, we can get at this quantity using the results we already
have, thanks to the following sneaky trick (due to Landau). If we
increment the momentum of each quasiparticle by a small amount
\(\delta\bk\), we can compute the new energy using our energy functional
Equation~\ref{eq-fdef}, along with a new distribution function

\begin{figure}[H]

{\centering \includegraphics[width=0.5\linewidth,height=\textheight,keepaspectratio]{../assets/FermiShift.png}

}

\caption{Shifting the Fermi sea to increase the momentum.}

\end{figure}%

\[
\begin{align}
N_s(\bk-\delta\bk) &=\theta(k_F-\abs{\bk-\delta\bk}) + n_{s}(\bk-\delta\bk)+\cdots\nonumber\\
&=\theta(k_F-\abs{\bk}) + n_s(\bk) + \delta(k_F-\abs{\bk})\hat{\bk}\cdot\delta\bk - \delta\bk \nabla_\bk n_{s}(\bk)+\cdots.
\end{align}
\]

Treating the last three terms as \(n_s(\bk)\), our excitation energy
changes to first order in \(\delta\bk\) by an amount

\[
\begin{align}
\Delta E &= \sum_{\bk,s} n_{s}(\bk)\delta\bk\cdot\nabla_\bk\varepsilon_s(\bk) \nonumber\\
&+\frac{1}{V}\sum_{\bk, s,\bk', s'} f_{s^{}s'}(\bk,\bk')n_{s}(\bk)\left[\delta(k_F-\abs{\bk'})\hat{\bk}'\cdot\delta\bk - \nabla_{\bk'}n_{s'}(\bk')\cdot\delta\bk\right].
\end{align}
\]

(In the first term I have integrated by parts.) On grounds of Galilean
invariance, however, we also know that this is

\begin{equation}\phantomsection\label{eq-gal}{
\Delta E = \frac{\mathbf{P}}{m}\cdot\delta\bk,
}\end{equation}

where the total momentum \(\mathbf{P}\) can be written

\[
\mathbf{P} = \sum_{\bk,s} \bk n_{s}(\bk).
\]

If Equation~\ref{eq-gal} holds for all \(n_s(\bk)\) and \(\delta\bk\),
we have (ignoring second order in \(n_s(\bk)\))

\[
\frac{\bk}{m} = \nabla_\bk\varepsilon_s(\bk) +  \sum_{s'}\int  f_{s^{}s'}(\bk,\bk')\delta(k_F-\abs{\bk'})\hat{\bk}' \frac{d\bk'}{(2\pi)^3}.
\]

Restricting ourselves to momenta close to the Fermi surface, and using
our definitions Equation~\ref{eq-fgdef}, Equation~\ref{eq-vfdef} and
Equation~\ref{eq-mdef} gives the relation

\[
\frac{\bk}{m} = \frac{\bk}{m_*} +\frac{1}{m} \int F(\phi) \bk' \frac{d\Omega_{\bk'}}{4\pi}.
\]

If we write \(\bk'=\cos\phi \bk + \sin\phi \bk_\perp\), with
\(\bk_\perp\cdot\bk=0\), this gives Landau's famous result

\[
\frac{1}{m} = \frac{1}{m_*} +\frac{1}{m} \int F(\phi) \cos\phi \frac{\sin\phi d\phi}{2}.
\]

For the \(F(\phi)\) that we found in Equation~\ref{eq-ffinal} from
second order perturbation theory, this gives the effective mass
correction

\[
\frac{m_*}{m} = 1 + \frac{1}{30\pi^4}(7\log 2 - 1)\left(mU_0k_\text{F}\right)^2+\cdots.
\]

(Use the substitution \(u=\sin\phi/2\) to do the integral.) Again, the
point is not the value that we've obtained, but the argument we used to
do so. In systems with strong interactions it's possible for the
effective mass to be very different from the bare mass: in the
\href{https://en.wikipedia.org/wiki/Heavy_fermion_material}{heavy
fermion materials} \(m_*/m\) can approach 1000! Despite being so far
from the noninteracting limit, Landau's picture of fermionic
quasiparticles still applies.

\chapter{Eigenstates in Perturbation Theory: What is a
Quasiparticle?}\label{eigenstates-in-perturbation-theory-what-is-a-quasiparticle}

So far we've focused on the energies of the excited states of the gas.
But what do these quasiparticle states \emph{look} like? In perturbation
theory at least, we can see fairly explicitly. At first order we have

\[
\ket{\mathbf{N}^{(1)}} = \sum_{\mathbf{N}'\neq \mathbf N}\frac{\braket{\mathbf{N'}}{H_\text{int}}{\mathbf{N}}}{E^{(0)}(\mathbf{N})-E^{(0)}(\mathbf{N}')}\ket{\mathbf{N}'}.
\]

Let's consider the Fermi sea ground state \(\ket{\text{FS}}\). What
states can appear in the above sum in this case? The only possibility is
that the interaction creates two particle-hole pairs out of the Fermi
sea, with total momentum zero.

\begin{figure}[H]

{\centering \includegraphics[width=0.5\linewidth,height=\textheight,keepaspectratio]{../assets/2ph.png}

}

\caption{Two particle-hole pairs created out of the Fermi sea.}

\end{figure}%

\[
\ket{0}=\ket{\text{FS}}+\text{two particle-hole pair states}+\cdots
\]

What about an excited state? Consider the state

\[
\adop_{\bk,s}\ket{\text{FS}},
\]

having momentum \(\bk\) and spin \(s\). When the interactions are
switched on, this state will be modified to a state we'll denote
\(\ket{\bk,s}\). At first order, two kinds of states can contribute to
the modifed state:

\begin{enumerate}
\def\labelenumi{\arabic{enumi}.}
\tightlist
\item
  States with a pair of particle-hole pairs, as before, but with the
  extra particle at \(\bk\). The coefficients of these states are
  \emph{the same} as for the corresponding states in the correction for
  \(\ket{0}\), as you should check.
\item
  States with a single particle-hole pair and with the extra particle
  moved from \(\bk\)\hspace{0pt} to another momentum.
\end{enumerate}

\begin{figure}[H]

{\centering \includegraphics[width=0.5\linewidth,height=\textheight,keepaspectratio]{../assets/phscatter.png}

}

\caption{Particle scatters, creating a particle-hole pair.}

\end{figure}%

Let's compare \(\ket{\bk,s}\) with \(\adop_{\bk,s}\ket{0}\), the state
obtained by creating a particle in the \emph{exact} ground state of the
problem. In first order perturbation theory, \(\ket{0}\) includes the
first kind of state above (2 particle-hole pair states). The states only
differ because of the contribution of the second kind. To first order,
the single quasiparticle state is therefore

\begin{equation}\phantomsection\label{eq-ph-peturb}{
\begin{align}
\ket{\bk,s} &= \sqrt{\frac{z_k}{\braket{0}{\aop_{\bk,s}\adop_{\bk,s}}{0}}}\adop_{\bk,s}\ket{0} \nonumber\\
&\qquad + \frac{U_0}{V}\sum_{\substack{\bk_1+\bk_2=\bk_3+\bk\\ s'}}\frac{\adop_{\bk_1,s}\adop_{\bk_2,s'}\aop_{\bk_3,s'}\ket{\text{FS}}}{\epsilon(\bk_1)+\epsilon(\bk_2)-\epsilon(\bk_3)-\epsilon(\bk)},
\end{align}
}\end{equation}

where \(\sqrt{z_k}\) is a normalization factor. As we go to successively
higher orders of perturbation theory, the quasiparticle state is
`dressed' with more particle-hole pairs. The quasiparticle retains the
conserved quantum numbers (momentum and spin in this case) of the
fermions of the noninteracting theory.

Normalizing Equation~\ref{eq-ph-peturb} gives

\begin{equation}\phantomsection\label{eq-z}{
z_\bk = 1 - \left(\frac{U_0}{V}\right)^2\sum_{\substack{\bk_1+\bk_2=\bk_3+\bk\\\abs{\bk_3}<k_\text{F},\abs{\bk_2},\abs{\bk_1}>k_\text{F}\\ s'}}\frac{1}{\left[\epsilon(\bk_1)+\epsilon(\bk_2)-\epsilon(\bk_3)-\epsilon(\bk)\right]^2}+\cdots.
}\end{equation}

This quantity can be interpreted in terms of the overlap of the single
quasiparticle state \(\ket{\bk,s}\) with \(\adop_{s,\bk}\ket{0}\).

\[
z_\bk = \frac{\abs{\braket{\bk,s}{\adop_{\bk,s}}{0}}^2}{\braket{0}{\aop_{\bk,s}\adop_{\bk,s}}{0}}
\]

A finite overlap -- evaluated for quasiparticles at the Fermi surface --
is a requirement for the Fermi liquid picture to hold. If it were to
vanish, any resemblance of the quasiparticle to a free fermion would
disappear with it!

Obviously evaluating the integrals in Equation~\ref{eq-z} is a challenge
involving a double integral over momentum, but I have it on good
authority (see Abrikosov, Gorkov, and Dzyaloshinski (2012), though they
don't give the details) that the answer is

\[
z_{\abs{\bk}=k_\text{F}} = 1 - \frac{(mUk_\text{F})^2}{8\pi^4}\left[\log 2 + \frac{1}{3}\right]
\]

This is also the occupation number of the original fermions
\(\braket{0}{\adop_{\bk,s}\aop_{\bk,s}}{0}\) (not the quasiparticles!)
just below the Fermi surface in the ground state (see
\href{../problem-sets/problem-set-3.qmd\#nmathbfk-in-the-ground-state}{Problem
Set 3}. There is a corresponding result just above. Even with
interactions, there is a finite step in the distribution function at the
Fermi surface.

\begin{figure}[H]

{\centering \includegraphics[width=0.7\linewidth,height=\textheight,keepaspectratio]{../assets/FermiJump.png}

}

\caption{Discontinutity in the ground state occupation number.}

\end{figure}%

\chapter{Collisions}\label{collisions}

The picture we have developed so far of eigenstates labelled by
quasiparticle occupations is actually a bit of an oversimplification.
When we apply the second order perturbation theory formula
Equation~\ref{eq-2nd} we have to omit degenerate states. It isn't really
enough to insist that the occupancies differ
\(\mathbf{N}'\neq \mathbf N\), because many states with different
occupancies have the same energy in the thermodynamic limit. From a
time-dependent point of view, the interaction can cause transitions
between these states, leading to the quasiparticle distribution changing
over time. The rate of these transitions can be described by the Fermi
golden rule.

\[
\Gamma_{\mathbf{N}\to\mathbf{N'}} = 2\pi \abs{\braket{\mathbf{N'}}{H_\text{int}}{\mathbf{N}}}^2 \delta(E(\mathbf{N})-E(\mathbf{N}'))
\]

By considering the volume of phase space available for the scattering of
a quasiparticle of energy \(\Delta\) above the Fermi surface, you should
be able to argue that the \emph{total} rate, obtained by integrating
over all possible final states, varies like \(\Delta^2\). This means
that at low energy (or temperature) such scattering is ineffective and
the Landau picture holds.

\begin{figure}[H]

{\centering \includegraphics[width=0.6\linewidth,height=\textheight,keepaspectratio]{../assets/Collision.png}

}

\caption{Vanishing phase space volume at low energies.}

\end{figure}%

\chapter{Appendix}\label{appendix}

\section{Rotationally invariant
formulation}\label{rotationally-invariant-formulation}

The definition Equation~\ref{eq-fdef} implied a certain quantization
axis for spin. To write things in an invariant way, we should think of
the occupation number \(\mathsf{N}(\bk)\) as a \(2\times 2\) matrix that
can describe an arbitrary spin orientation, with elements

\[
\mathsf{N}(\bk)=\begin{pmatrix}
N_{\uparrow\uparrow}(\bk) & N_{\uparrow\downarrow}(\bk) \\
N_{\downarrow\uparrow}(\bk) & N_{\downarrow\downarrow}(\bk).
\end{pmatrix}
\]

The \(f\)-function then has \emph{four} spin indices

\[
\frac{1}{2V}\sum_{\bk, s_1,s_2,\bk', s_3,s_4} f_{s_1s_2,s_3s_4}(\bk,\bk')n_{s_1s_2}(\bk)n_{s_3s_4}(\bk').
\]

When the occupation numbers matrix is diagonal, we identify
\(n_{\uparrow\uparrow}(\bk) = n_\uparrow(\bk)\) and
\(n_{\downarrow\downarrow}(\bk) = n_\downarrow(\bk)\).

When rotated, \(\mathsf{N}\) transforms by conjugation by a matrix
\(\mathsf{U}\in SU(2)\) \[
\mathsf{N} \longrightarrow \mathsf{U}\mathsf{N} \mathsf{U}^\dagger.
\]

There are thus two rotationally invariant contributions to
\(f_{s_1s_2,s_3s_4}(\bk,\bk')\), with index structure
\(\delta_{s_1s_2}\delta_{s_3s_4}\) and
\(\delta_{s_1s_4}\delta_{s_2s_3}\). Alternatively, we can use the
identity involving Pauli matrices that we met in
\href{more-second-quantization.qmd}{Lecture 6}

\[
\delta_{s_1s_4}\delta_{s_2s_3} = \frac{1}{2}\left[\boldsymbol{\sigma}_{s_1s_2}\cdot \boldsymbol{\sigma}_{s_3s_4} + \delta_{s_1s_2}\delta_{s_3s_4}\right].
\]

This shows that the two rotationally invariant contributions can be
taken to have the index structure \(\delta_{s_1s_2}\delta_{s_3s_4}\) and
\(\boldsymbol{\sigma}_{s_1s_2}\cdot \boldsymbol{\sigma}_{s_3s_4}\).
Comparing with Equation~\ref{eq-f-rewrite} in the diagonal case gives

\[
\nu(E_\text{F})f_{s_1s_2,s_3s_4}(\bk,\bk') = F(\phi) \delta_{s_1s_2}\delta_{s_3s_4} + G(\phi)\boldsymbol{\sigma}_{s_1s_2}\cdot\boldsymbol{\sigma}_{s_3s_4}.
\]

\phantomsection\label{refs}
\begin{CSLReferences}{1}{0}
\bibitem[\citeproctext]{ref-abrikosov2012methods}
Abrikosov, Aleksei Alekseevich, Lev Petrovich Gorkov, and Igor
Ekhielevich Dzyaloshinski. 2012. \emph{Methods of Quantum Field Theory
in Statistical Physics}. Courier Corporation.

\bibitem[\citeproctext]{ref-landau1980statistical}
Landau, Lev Davidovich, Evgenii Mikhailovich Lifshitz, Evgenii
Mikhailovich Lifshitz, and LP Pitaevskii. 1980. \emph{Statistical
Physics: Theory of the Condensed State}. Vol. 9. Butterworth-Heinemann.

\end{CSLReferences}


\backmatter


\end{document}
