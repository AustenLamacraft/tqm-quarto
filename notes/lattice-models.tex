% Options for packages loaded elsewhere
% Options for packages loaded elsewhere
\PassOptionsToPackage{unicode}{hyperref}
\PassOptionsToPackage{hyphens}{url}
%
\documentclass[
  a4paper,
]{scrbook}
\usepackage{xcolor}
\usepackage{amsmath,amssymb}
\setcounter{secnumdepth}{5}
\usepackage{iftex}
\ifPDFTeX
  \usepackage[T1]{fontenc}
  \usepackage[utf8]{inputenc}
  \usepackage{textcomp} % provide euro and other symbols
\else % if luatex or xetex
  \usepackage{unicode-math} % this also loads fontspec
  \defaultfontfeatures{Scale=MatchLowercase}
  \defaultfontfeatures[\rmfamily]{Ligatures=TeX,Scale=1}
\fi
\usepackage{lmodern}
\ifPDFTeX\else
  % xetex/luatex font selection
\fi
% Use upquote if available, for straight quotes in verbatim environments
\IfFileExists{upquote.sty}{\usepackage{upquote}}{}
\IfFileExists{microtype.sty}{% use microtype if available
  \usepackage[]{microtype}
  \UseMicrotypeSet[protrusion]{basicmath} % disable protrusion for tt fonts
}{}
\makeatletter
\@ifundefined{KOMAClassName}{% if non-KOMA class
  \IfFileExists{parskip.sty}{%
    \usepackage{parskip}
  }{% else
    \setlength{\parindent}{0pt}
    \setlength{\parskip}{6pt plus 2pt minus 1pt}}
}{% if KOMA class
  \KOMAoptions{parskip=half}}
\makeatother
% Make \paragraph and \subparagraph free-standing
\makeatletter
\ifx\paragraph\undefined\else
  \let\oldparagraph\paragraph
  \renewcommand{\paragraph}{
    \@ifstar
      \xxxParagraphStar
      \xxxParagraphNoStar
  }
  \newcommand{\xxxParagraphStar}[1]{\oldparagraph*{#1}\mbox{}}
  \newcommand{\xxxParagraphNoStar}[1]{\oldparagraph{#1}\mbox{}}
\fi
\ifx\subparagraph\undefined\else
  \let\oldsubparagraph\subparagraph
  \renewcommand{\subparagraph}{
    \@ifstar
      \xxxSubParagraphStar
      \xxxSubParagraphNoStar
  }
  \newcommand{\xxxSubParagraphStar}[1]{\oldsubparagraph*{#1}\mbox{}}
  \newcommand{\xxxSubParagraphNoStar}[1]{\oldsubparagraph{#1}\mbox{}}
\fi
\makeatother


\usepackage{longtable,booktabs,array}
\usepackage{calc} % for calculating minipage widths
% Correct order of tables after \paragraph or \subparagraph
\usepackage{etoolbox}
\makeatletter
\patchcmd\longtable{\par}{\if@noskipsec\mbox{}\fi\par}{}{}
\makeatother
% Allow footnotes in longtable head/foot
\IfFileExists{footnotehyper.sty}{\usepackage{footnotehyper}}{\usepackage{footnote}}
\makesavenoteenv{longtable}
\usepackage{graphicx}
\makeatletter
\newsavebox\pandoc@box
\newcommand*\pandocbounded[1]{% scales image to fit in text height/width
  \sbox\pandoc@box{#1}%
  \Gscale@div\@tempa{\textheight}{\dimexpr\ht\pandoc@box+\dp\pandoc@box\relax}%
  \Gscale@div\@tempb{\linewidth}{\wd\pandoc@box}%
  \ifdim\@tempb\p@<\@tempa\p@\let\@tempa\@tempb\fi% select the smaller of both
  \ifdim\@tempa\p@<\p@\scalebox{\@tempa}{\usebox\pandoc@box}%
  \else\usebox{\pandoc@box}%
  \fi%
}
% Set default figure placement to htbp
\def\fps@figure{htbp}
\makeatother





\setlength{\emergencystretch}{3em} % prevent overfull lines

\providecommand{\tightlist}{%
  \setlength{\itemsep}{0pt}\setlength{\parskip}{0pt}}



 


\makeatletter
\@ifpackageloaded{tcolorbox}{}{\usepackage[skins,breakable]{tcolorbox}}
\@ifpackageloaded{fontawesome5}{}{\usepackage{fontawesome5}}
\definecolor{quarto-callout-color}{HTML}{909090}
\definecolor{quarto-callout-note-color}{HTML}{0758E5}
\definecolor{quarto-callout-important-color}{HTML}{CC1914}
\definecolor{quarto-callout-warning-color}{HTML}{EB9113}
\definecolor{quarto-callout-tip-color}{HTML}{00A047}
\definecolor{quarto-callout-caution-color}{HTML}{FC5300}
\definecolor{quarto-callout-color-frame}{HTML}{acacac}
\definecolor{quarto-callout-note-color-frame}{HTML}{4582ec}
\definecolor{quarto-callout-important-color-frame}{HTML}{d9534f}
\definecolor{quarto-callout-warning-color-frame}{HTML}{f0ad4e}
\definecolor{quarto-callout-tip-color-frame}{HTML}{02b875}
\definecolor{quarto-callout-caution-color-frame}{HTML}{fd7e14}
\makeatother
\makeatletter
\@ifpackageloaded{caption}{}{\usepackage{caption}}
\AtBeginDocument{%
\ifdefined\contentsname
  \renewcommand*\contentsname{Table of contents}
\else
  \newcommand\contentsname{Table of contents}
\fi
\ifdefined\listfigurename
  \renewcommand*\listfigurename{List of Figures}
\else
  \newcommand\listfigurename{List of Figures}
\fi
\ifdefined\listtablename
  \renewcommand*\listtablename{List of Tables}
\else
  \newcommand\listtablename{List of Tables}
\fi
\ifdefined\figurename
  \renewcommand*\figurename{Figure}
\else
  \newcommand\figurename{Figure}
\fi
\ifdefined\tablename
  \renewcommand*\tablename{Table}
\else
  \newcommand\tablename{Table}
\fi
}
\@ifpackageloaded{float}{}{\usepackage{float}}
\floatstyle{ruled}
\@ifundefined{c@chapter}{\newfloat{codelisting}{h}{lop}}{\newfloat{codelisting}{h}{lop}[chapter]}
\floatname{codelisting}{Listing}
\newcommand*\listoflistings{\listof{codelisting}{List of Listings}}
\makeatother
\makeatletter
\makeatother
\makeatletter
\@ifpackageloaded{caption}{}{\usepackage{caption}}
\@ifpackageloaded{subcaption}{}{\usepackage{subcaption}}
\makeatother
\usepackage{bookmark}
\IfFileExists{xurl.sty}{\usepackage{xurl}}{} % add URL line breaks if available
\urlstyle{same}
\hypersetup{
  pdftitle={Lattice Models},
  hidelinks,
  pdfcreator={LaTeX via pandoc}}


\title{Lattice Models}
\author{}
\date{}
\begin{document}
\frontmatter
\maketitle


\mainmatter
\providecommand{\cE}{\mathcal{E}}
\providecommand{\cH}{\mathcal{H}}
\providecommand{\cN}{\mathcal{N}}
\providecommand{\cO}{\mathcal{O}}
\providecommand{\cP}{\mathcal{P}}
\providecommand{\br}{\mathbf{r}}
\providecommand{\bp}{\mathbf{p}}
\providecommand{\bk}{\mathbf{k}}
\providecommand{\bq}{\mathbf{q}}
\providecommand{\bv}{\mathbf{v}}
\providecommand{\brN}{\br_1, \ldots, \br_N}
\providecommand{\xN}{x_1, \ldots, x_N}
\providecommand{\zN}{z_1, \ldots, z_N}
\providecommand{\pop}{\psi^{\vphantom{\dagger}}}
\providecommand{\pdop}{\psi^\dagger}
\providecommand{\Pop}{\Psi^{\vphantom{\dagger}}}
\providecommand{\Pdop}{\Psi^\dagger}
\providecommand{\Phop}{\Phi^{\vphantom{\dagger}}}
\providecommand{\Phdop}{\Phi^\dagger}
\providecommand{\phop}{\phi^{\vphantom{\dagger}}}
\providecommand{\phdop}{\phi^\dagger}
\providecommand{\aop}{a^{\vphantom{\dagger}}}
\providecommand{\adop}{a^\dagger}
\providecommand{\bop}{b^{\vphantom{\dagger}}}
\providecommand{\bdop}{b^\dagger}
\providecommand{\cop}{c^{\vphantom{\dagger}}}
\providecommand{\cdop}{c^\dagger}
\providecommand{\alop}{\alpha^{\vphantom{\dagger}}}
\providecommand{\aldop}{\alpha^\dagger}
\providecommand{\bra}[1]{\langle{#1}\rvert}
\providecommand{\ket}[1]{\lvert{#1}\rangle}
\providecommand{\inner}[2]{\langle{#1}\rvert #2 \rangle}
\providecommand{\braket}[3]{\langle{#1}\rvert #2 \lvert #3 \rangle}
\providecommand{\sgn}{\mathrm{sgn}}
\providecommand{\abs}[1]{\lvert{#1}\rvert}
\providecommand{\Nop}{\mathsf{N}^{\vphantom{\dagger}}}

\providecommand{\tr}{\mathrm{tr}}
\providecommand{\E}{\mathbb{E}}

Models defined on discrete sites --- so-called \textbf{tight binding
models} --- provide a conceptually simple way to think about the effects
of strong interactions between particles. The phenomenology of these
models is central to many current avenues of research in condensed
matter, whether in solid state or atomic physics.

\begin{center}\rule{0.5\linewidth}{0.5pt}\end{center}

\chapter{Tight Binding Models}\label{tight-binding-models}

A typical many body Hamiltonian consists of kinetic energy and
interaction terms. We haven't yet had much to say about the situation
where the particles additionally feel a potential that could arise from
the crystal lattice, or impurity atoms, or both. In this lecture we will
be concerned with systems in periodic potentials, so that the
noninteracting part of the Hamiltonian is (taking the 1D case for
simplicity)

\[
H = \sum_{j=1}^N \left[-\frac{1}{2m}\partial_i^2 +V(x_i)\right] = \int \left[\frac{1}{2m}\partial_x\pdop\partial_x\pop + V(x)\pdop\pop\right] dx,
\]

with \(V(x+a)=V(x)\). As you know,
\href{https://en.wikipedia.org/wiki/Bloch_wave}{Bloch's theorem} tells
us that the eigenstates are labelled by a continuous index \(k\)
(\textbf{crystal momentum}) and discrete index \(n\) (\textbf{band
index}) and have the form

\begin{equation}\phantomsection\label{eq-bloch}{
\psi_{k,n}(x) = e^{ikx} \varphi_{k,n}(x),
}\end{equation}

where \(\varphi_{k,n}\) is periodic. \(k\) lies in the \textbf{Brillouin
zone} \((-\pi/a,\pi/a]\). The eigenvalues \(E_n(k)\) give the
\textbf{energy bands}.

We are going to be concerned with the case where the lattice potential
is very strong, so that the wavefunctions -- at least in the lowest
bands that we assume are those occupied -- are highly localized. The
wavefunctions become very small between the minima of the potential. We
will see that in this limit we can introduce operators \(\adop_j\),
\(\aop_j\) describing particles in these localized states, and that the
coupling between neighbouring sites can be captured by the \textbf{tight
binding Hamiltonian}

\begin{equation}\phantomsection\label{eq-tb}{
H_t = -t \sum_{j} \left[\adop_j\aop_{j+1}+\adop_{j+1}\aop_j\right],
}\end{equation}

which will play the role of kinetic energy in our models. In fact, we've
already met such a description, when we described a magnon propagating
in a spin chain in
\href{\%5Bref\%20\%5B\%5D\%7B.quarto-shortcode__-param\%20data-is-shortcode=\%221\%22\%20data-value=\%22spin-models\%22\%20data-raw=\%22\%22spin-models\%22\%22\%7D\%20\%5D\%7B.quarto-shortcode__\%20data-is-shortcode=\%221\%22\%20data-raw=\%22\%7B\%7B\%3C\%20ref\%20\%22spin-models\%22\%20\%3E\%7D\%7D\%22\%7D}{Lecture
4}.

\section{Physical Motivation}\label{physical-motivation}

This subsection is to shed some light on the origin of
Equation~\ref{eq-tb}. Feel free to jump to the conclusion.

Let's think about the form of the Bloch states in a deep 1D lattice in
more detail. You know that in a stationary state, the current \(j(x)\)
is conserved

\begin{equation}\phantomsection\label{eq-j}{
j = -\frac{i}{2m}\left[\psi^*(\partial_x\psi^{}) - (\partial_x\psi^*)\psi^{}\right].
}\end{equation}

Representing \(\psi(x)=\sqrt{\rho(x)}e^{i\theta(x)}\), this can be
written

\[
j = \frac{\rho\partial_x\theta}{m}.
\]

The Bloch states at non-zero \(k\) carry a current. As we've said, away
from the minima of the lattice potential, the amplitude of the
wavefunction is very small. Thus if \(j\) is constant,
\(\partial_x\theta\) must be large. Essentially all of the change in the
phase of the wavefunction happens in these regions. Where the
wavefunction is larger, its phase is barely changing.

To describe this situation more quantitatively, we use the WKB
approximation, which consists in writing the wavefunction in the form

\[
\psi_\text{WKB}(x) = \frac{\alpha}{\sqrt{k(x)}}\exp\left(i\int^ x k(x')dx'\right)+ \frac{\beta}{\sqrt{k(x)}}\exp\left(-i\int^ x k(x')dx'\right),
\]

where \(k(x) = \sqrt{2m(E-V(x))}\). In fact, we want to describe the
part of the wavefunction with real exponents, where
\(k(x) = i\kappa(x)\) because \(V(x)>E\). Substitution into the current
Equation~\ref{eq-j} gives

\begin{equation}\phantomsection\label{eq-jwkb}{
j = \frac{1}{m}\text{Im}\, \alpha^*\beta^{}.
}\end{equation}

\begin{figure}[H]

{\centering \includegraphics[width=0.6\linewidth,height=\textheight,keepaspectratio]{../assets/WKB.png}

}

\caption{The Bloch states in the WKB picture.}

\end{figure}%

\(\alpha\) is the amplitude of the wavefunction in the well on the left,
and \(\beta\) is the amplitude on the right. By periodicity of the Bloch
state, it is only their phase that differs. Call this phase change
\(\theta\). In terms of the Bloch states Equation~\ref{eq-bloch}
\(\theta=ka\). Then Equation~\ref{eq-jwkb} says

\begin{equation}\phantomsection\label{eq-j-sin}{
j = \frac{|\alpha|^2}{m}\sin\theta.
}\end{equation}

Now the Bloch function \(\varphi_k(x)\) satisfies the Schrödinger
equation (we drop the band index)

\[
\left[-\frac{1}{2m}\left(\partial_x + ik\right)^2 + V(x)\right]\varphi_k(x) = E_k \varphi_k(x),
\]

where a vector potential has arisen from the phase factor \(e^{ikx}\).
By considering a small change \(\Delta k\) as a perturbation we can find

\[
\frac{\partial E_k}{\partial k} = ja,
\]

or

\[
\frac{\partial E_k}{\partial \theta} = j.
\]

Together with Equation~\ref{eq-j-sin} we conclude that the band has the
form

\[
E(k)-E(0) = 2t\left[1-\cos ka\right],
\]

with some characteristic energy \(t\). Thus we have found

\begin{enumerate}
\def\labelenumi{\arabic{enumi}.}
\item
  The Bloch states look like a superposition of localized states, with
  the phase changing abruptly between sites.
\item
  The lowest band is sinusoidal.
\end{enumerate}

\begin{tcolorbox}[enhanced jigsaw, leftrule=.75mm, bottomrule=.15mm, colbacktitle=quarto-callout-tip-color!10!white, opacitybacktitle=0.6, colback=white, rightrule=.15mm, colframe=quarto-callout-tip-color-frame, toprule=.15mm, left=2mm, titlerule=0mm, title=\textcolor{quarto-callout-tip-color}{\faLightbulb}\hspace{0.5em}{Check}, arc=.35mm, breakable, bottomtitle=1mm, toptitle=1mm, opacityback=0, coltitle=black]

Confirm that both these features are captured by the tight binding model
Equation~\ref{eq-tb}.

\end{tcolorbox}

\chapter{Bose Hubbard Model}\label{bose-hubbard-model}

The simplest interacting tight binding model that we can write down is
the celebrated \textbf{Hubbard model}.

\[
H = H_t + H_U = -t \sum_{\langle j\,k\rangle}  \left[\adop_j\aop_{k}+\adop_{k}\aop_j\right] + \frac{U}{2}\sum_j \Nop_j(\Nop_j-1),
\]

where \(\Nop_j=\adop_j\aop_j\) is the number operator for site \(j\),
and the sum in the first term is over all nearest neighbour pairs on
some lattice (e.g.~square or cubic). The on-site interaction term
\(\Nop_j(\Nop_j-1)\) is just what we have been writing as
\(\adop_j\adop_j\aop_j\aop_j\) previously. In 1D, you can think of this
as the lattice version of the Lieb--Liniger model, in that in involves a
kinetic term (sometimes called the \textbf{hopping term}) and a
short-ranged interaction.

The Hubbard model was first introduced to describe electrons in solids.
In that case the particles are fermions, and we have to deal with their
spin. We'll come to this \textbf{Fermi Hubbard model} shortly. In this
section we are going to discuss the more straightforward \textbf{Bose
Hubbard} model introduced in \{\% cite Fisher:1989aa \%\}. These authors
had in mind applications to liquid He\(^{4}\) on a substrate or granular
superconductors in which the Cooper pairs (more on those later)
approximate interacting bosons. However, ten years after its invention,
the model found its natural home in the description of bosonic atoms
confined to optical lattices.

\section{The Mott State}\label{the-mott-state}

In analyzing a new model we always begin by asking what happens when
each of the terms in the Hamiltonian dominate the others. This allows us
to get the lie of the land, and think about how these different limits
may fit together.

For the Bose--Hubbard model, we start by taking \(U/t\to\infty\) The
eigenstates are then states of definite occupation number, with energy

\[
E(\mathbf{N}) = \frac{U}{2} \sum_j N_j(N_j-1).
\]

To find the ground state at fixed particle number, we should try filling
the \(N_\text{sites}\) sites as uniformly as possible. This is easy if
the filling \(\nu \equiv N_{\text{particles}}/N_\text{sites}\) is
integer. Otherwise, denote by \(\lfloor \nu\rfloor\) the largest integer
less than \(\nu\), and \(\lceil \nu\rceil\) the smallest integer larger
than \(\nu\). Then the number of sites with occupancy
\(\lceil \nu\rceil\) is
\(N_\text{sites}\left(\nu - \lfloor \nu\rfloor\right)\). The ground
state energy has a piecewise linear dependence on \(\nu\)

\[
\frac{E_0}{N_\text{sites}} = \left(\nu - \lfloor \nu\rfloor\right)e(\lceil \nu\rceil) + \left(\lceil \nu\rceil - \nu\right)e(\lfloor \nu\rfloor),
\]

where \(e(N) = \frac{U}{2}N(N-1)\). As the result the chemical potential
\(\mu = \frac{\partial E_0}{\partial N_\text{particles}}\) is piecewise
constant, with jumps occurring when \(\nu\) is integer:

\[
\mu = e(\lceil \nu\rceil) -e(\lfloor \nu\rfloor)=U\lfloor \nu\rfloor,
\]

\begin{figure}[H]

{\centering \includegraphics[width=0.6\linewidth,height=\textheight,keepaspectratio]{../assets/steps.png}

}

\caption{Energy and chemical potential vs.~filling \(\nu\).}

\end{figure}%

If we think about varying the chemical potential instead, all values
between \(UN\) and \(U(N-1)\) correspond to integer filling \(\nu=N\).
The states of integer filling are named \textbf{Mott states}, after
\href{https://en.wikipedia.org/wiki/Nevill_Francis_Mott}{Nevill Mott}.
Mott's realization was that when interactions dominate the hopping, the
formation of such states can cause insulating behaviour, even when band
theory implies a system should be a metal.

\section{The Effect of Hopping}\label{the-effect-of-hopping}

For these states to be physically significant, they must survive when we
turn \(H_t\) back on. If the hopping is small, we can think of it as a
perturbation. The unperturbed eigenstates of the model have a fixed
occupation \(N_j\) on site \(j\)

\[
\ket{\mathbf{N}} = \bigotimes_j \ket{N_j}_j.
\]

Applying \(H_t\) to such a state gives a superposition of states, each
with one particle moved from one site to an adjacent site.

\begin{tcolorbox}[enhanced jigsaw, leftrule=.75mm, bottomrule=.15mm, colbacktitle=quarto-callout-tip-color!10!white, opacitybacktitle=0.6, colback=white, rightrule=.15mm, colframe=quarto-callout-tip-color-frame, toprule=.15mm, left=2mm, titlerule=0mm, title=\textcolor{quarto-callout-tip-color}{\faLightbulb}\hspace{0.5em}{Check}, arc=.35mm, breakable, bottomtitle=1mm, toptitle=1mm, opacityback=0, coltitle=black]

How is a Mott state (all sites with the same occupation) changed by
\(H_t\) in first order perturbation theory?

\end{tcolorbox}

Let's now consider a Mott state of filling \(\nu=N\) with one extra
particle added. We now have \(N_\text{sites}\) degenerate ground states
when \(t=0\), corresponding to placing the extra particle on each
lattice site. \(H_t\) mixes these states together: we have a problem of
degenerate perturbation theory. All other states are separated from
these lowest states by energies of order \(U\).

The states in the ground state multiplet are

\[
\ket{i,+}\equiv \frac{\adop_i}{\sqrt{N+1}}\bigotimes_{j} \ket{N}_j.
\]

Evidently, only states corresponding to adjacent \(j\) are coupled by
\(H_t\). These matrix elements are

\[
\braket{j}{H_t}{k} = -t(N+1).
\]

Thus within the ground state multiplet \(H_t\) corresponds to a tight
binding model

\begin{equation}\phantomsection\label{eq-tbp}{
H_t\rvert_{+} = -t(N+1) \sum_{\langle j\,k\rangle} \left[\ket{j,+}\bra{k,+}+\text{h.c.}\right].
}\end{equation}

This is a very simple picture: the only many body effect is the factor
of \(N\) due to Bose statistics. The splitting of the degenerate states
by \(H_t\) is then given by the tight binding dispersion

\begin{equation}\phantomsection\label{eq-pband}{
\omega_+(\boldsymbol{\eta}) = -2t(N+1)\sum_{n=1}^d\cos\eta_n
}\end{equation}

(in \(d\)-dimensions). We could alternatively \emph{remove} a particle
from the Mott state

\[
\ket{i,-}\equiv \frac{\aop_i}{\sqrt{N}}\bigotimes_{j} \ket{N}_j.
\]

Within these states, \(H_t\) takes the form

\begin{equation}\phantomsection\label{eq-tbh}{
H_t\rvert_{-}=-tN \sum_{\langle j\,k\rangle} \left[\ket{j,-}\bra{k,-}+\text{h.c.}\right],
}\end{equation}

with a spectrum

\[
\omega_-(\boldsymbol{\eta}) = -2tN\sum_{n=1}^d\cos\eta_n.
\]

Let's see how these considerations change the picture of the previous
subsection. We introduce the grand canonical Hamiltonian

\[
\cH_\mu = H - \mu N_\text{particles},
\]

and consider the ground state as a function of \(\mu\). At \(t=0\) the
energies of the Mott states with filling \(\nu=N\) are

\[
\frac{\cE^{(N)}_\mu}{N_\text{sites}} = \frac{U}{2}N(N-1)-\mu N.
\]

\(\cE^{(N)}_\mu\) and \(\cE^{(N+1)}_\mu\) become degenerate when
\(\mu=UN\) for \(t=0\). Let's compare this with the ground state with
one extra particle on top of the \(N\) Mott state. This state
corresponds to a particle at the bottom of the tight binding band. From
Equation~\ref{eq-pband}, we see that the ground state energy of the
tight binding model is \(-2td(N+1)\). Together with extra energy of
interaction we have overall

\[
\cE^{(N)}_\mu + UN - \mu -2dt(N+1).
\]

We see that for \(t>\frac{UN-\mu}{2d(N+1)}\) the state with an extra
particle actually has a \emph{lower} energy: the Mott state is not the
ground state. Similarly, the energy of the ground state with one `hole'
in the \(N+1\) Mott state is

\[
\cE^{(N+1)}_\mu - UN + \mu -2dt(N+1).
\]

Introducing a hole is thus favoured for \(t>\frac{\mu-UN}{2d(N+1)}\).

\begin{figure}[H]

{\centering \includegraphics[width=0.6\linewidth,height=\textheight,keepaspectratio]{../assets/MottPerturb.png}

}

\caption{Comparing the energies of the Mott states with single particle
or hole states. At \(t=0\) the red line is the absolute ground state.
For nonzero \(t\) the dashed blue line corresponds to non-integer
filling.}

\end{figure}%

This analysis applies only at small \(t/U\). What happens in the regions
where the Mott states are not the ground states, and there are excess
bosonic particles or holes that are free to move? If we let
\(t/U\to\infty\) we have a \textbf{Bose condensate}: all the particles
can sit in the \(\eta=0\) Bloch state. When interactions are finite but
small, we will see in the next lecture that the result is a
\textbf{superfluid}. The boundaries that we have have found can be
connected (drawing freehand -- see
\href{\%5Bref\%20\%5B\%5D\%7B.quarto-shortcode__-param\%20data-is-shortcode=\%221\%22\%20data-value=\%22problem-set-2\#mean-field-for-bosehubbard\%22\%20data-raw=\%22\%22problem-set-2\#mean-field-for-bosehubbard\%22\%22\%7D\%20\%5D\%7B.quarto-shortcode__\%20data-is-shortcode=\%221\%22\%20data-raw=\%22\%7B\%7B\%3C\%20ref\%20\%22problem-set-2\#mean-field-for-bosehubbard\%22\%20\%3E\%7D\%7D\%22\%7D}{Problem
Set 2} for a variational approach) to give the following phase diagram
for the ground state of the Bose--Hubbard model.

\begin{figure}[H]

{\centering \includegraphics[width=0.6\linewidth,height=\textheight,keepaspectratio]{../assets/BHPhase.png}

}

\caption{Ground state phase diagram of the Bose--Hubbard model.}

\end{figure}%

Note the diminishing size of the Mott lobes, a consequence of the
enhanced hopping in the effective tight binding models
Equation~\ref{eq-tbp} and Equation~\ref{eq-tbh} as we go to higher
filling.

\begin{figure}[H]

{\centering \pandocbounded{\includegraphics[keepaspectratio]{../assets/mott_state_cake.jpg}}

}

\caption{With a trap potential as well as a lattice, moving radially
outwards corresponds to moving down a vertical slice through the phase
diagram, producing this distinctive `wedding cake' structure. Successive
Mott states are separated by superfluid regions. Source: Cheng Chin,
University of Chicago.}

\end{figure}%

\chapter{Fermi Hubbard Model}\label{fermi-hubbard-model}

Now we turn to the case of fermions, the context in which the Hubbard
model was originally introduced. Allowing for spin, the model is usually
written.

\[
H=-t \sum_{\substack{\langle j\,k\rangle\\ s=\uparrow,\downarrow}}  \left[\adop_{j,s}\aop_{k,s}+\adop_{k,s}\aop_{j,s}\right] + U\sum_j N_\uparrow N_\downarrow,
\]

We could of course add spin to the Bose Hubbard model, but it's still
interesting without it. By contrast, we need spin here to have an
interacting model. It's hard to overstate the importance of this model
in condensed matter physics because of the role that it has played in
attempts to understand the high-temperature superconducting materials
known as \textbf{cuprates}. The 2D model has long been argued to capture
the physics of strong correlations in the CuO\(_2\) layers that form the
backbone of these materials. `Capture the physics' would ideally mean
that the ground state of the model is superconducting. This is still a
controversial issue: some believe that the Hubbard model suffices, if
only we could learn enough about its behaviour; others that it is
missing some ingredient -- other bands, coupling between layers, phonons
-- that is vital to superconductivity.

Relatively little is known \emph{for sure} about the Hubbard model,
except in 1D, where it can be solved exactly using the Bethe ansatz. You
may be wondering why it's so much harder than the Bose case. Let's find
out\ldots{}

\section{Two Sites, Two Fermions}\label{two-sites-two-fermions}

As in the Bose case, we start by thinking about \(U/t\to\infty\). In the
limit we get Mott states: only three this time, corresponding to 0, 1,
or 2 particles per site. Two particles on a site are described by
\(\adop_{j,s}\adop_{j,s'}\ket{\text{VAC}}\), which is antisymmetric in
the spin indices due to anticommutation of the creation operators, and
therefore describes a spin singlet
\(\frac{1}{\sqrt{2}}\left[\ket{\uparrow}\ket{\downarrow}-\ket{\downarrow}\ket{\uparrow}\right]\).
When we have only 1 per site (We call this \textbf{half filling}), we
can have either spin. Thus the \(\nu=1\) Mott state is \emph{massively}
degenerate, with a ground state multiplet consisting of
\(2^{N_\text{sites}}\) possible spin configurations. \(U/t\to\infty\) is
therefore a rather singular limit, and to understand the true ground
state at large \(U\) we'll need to work bit harder.

Start by thinking about two sites and two particles. There are 6 states
altogether in the Hilbert space.

\[
\begin{aligned}
\adop_{1,\uparrow}\adop_{1,\downarrow} \ket{\text{VAC}},\quad\adop_{2,\uparrow}\adop_{2,\downarrow} \ket{\text{VAC}}\\
\adop_{1,s}\adop_{2,s'} \ket{\text{VAC}},\quad s,s'=\uparrow,\downarrow.
\end{aligned}
\]

The top two states have energy \(U\) when \(t=0\); the bottom 4 have
energy 0.

\begin{tcolorbox}[enhanced jigsaw, leftrule=.75mm, bottomrule=.15mm, colbacktitle=quarto-callout-tip-color!10!white, opacitybacktitle=0.6, colback=white, rightrule=.15mm, colframe=quarto-callout-tip-color-frame, toprule=.15mm, left=2mm, titlerule=0mm, title=\textcolor{quarto-callout-tip-color}{\faLightbulb}\hspace{0.5em}{Check}, arc=.35mm, breakable, bottomtitle=1mm, toptitle=1mm, opacityback=0, coltitle=black]

How does the degeneracy of these states get lifted at finite \(t\)? Try
writing down the Hamiltonian restricted to these states.

\begin{quote}
\textbf{Solution}

The first thing to note is that the Hamiltonian \(H_t\) has no effect on
the states \(\adop_{1,\uparrow}\adop_{2,\uparrow} \ket{\text{VAC}}\) and
\(\adop_{1,\downarrow}\adop_{2,\downarrow} \ket{\text{VAC}}\), because
it is not possible to move two fermions with the same spin to the same
site. Thus we need to consider only the four states

\[
\begin{aligned}
\ket{1}&=\adop_{1,\uparrow}\adop_{1,\downarrow} \ket{\text{VAC}}\\
\ket{2}&=\adop_{2,\uparrow}\adop_{2,\downarrow} \ket{\text{VAC}}\\
\ket{3}&=\adop_{1,\uparrow}\adop_{2,\downarrow} \ket{\text{VAC}}\\
\ket{4}&=\adop_{1,\downarrow}\adop_{2,\uparrow} \ket{\text{VAC}}\\
\end{aligned}
\]

\(H_t\) connects \(\ket{1}\) and \(\ket{2}\) with \(\ket{3}\) and
\(\ket{4}\). We compute the matrix elements

\[
\begin{aligned}
\braket{1}{H_t}{3} &= \braket{\text{VAC}}{\aop_{1,\downarrow}\aop_{1,\uparrow} H_t \adop_{1,\uparrow}\adop_{2,\downarrow}}{\text{VAC}}\\
&=-t\braket{\text{VAC}}{\aop_{1,\downarrow}\aop_{1,\uparrow} \adop_{1,\downarrow}\aop_{2,\downarrow} \adop_{1,\uparrow}\adop_{2,\downarrow}}{\text{VAC}}\\
&=t\\
\braket{2}{H_t}{3} &= \braket{\text{VAC}}{\aop_{2,\downarrow}\aop_{2,\uparrow} H_t \adop_{1,\uparrow}\adop_{2,\downarrow}}{\text{VAC}}\\
&=-t\braket{\text{VAC}}{\aop_{2,\downarrow}\aop_{2,\uparrow} \adop_{2,\uparrow}\aop_{1,\uparrow} \adop_{1,\uparrow}\adop_{2,\downarrow}}{\text{VAC}}\\
&=-t\\
\braket{1}{H_t}{4} &= \braket{\text{VAC}}{\aop_{1,\downarrow}\aop_{1,\uparrow} H_t \adop_{1,\downarrow}\adop_{2,\uparrow}}{\text{VAC}}\\
&=-t\braket{\text{VAC}}{\aop_{1,\downarrow}\aop_{1,\uparrow} \adop_{1,\uparrow}\aop_{2,\uparrow} \adop_{1,\downarrow}\adop_{2,\uparrow}}{\text{VAC}}\\
&=t\\
\braket{2}{H_t}{4} &= \braket{\text{VAC}}{\aop_{2,\downarrow}\aop_{2,\uparrow} H_t \adop_{1,\downarrow}\adop_{2,\uparrow}}{\text{VAC}}\\
&=-t\braket{\text{VAC}}{\aop_{2,\downarrow}\aop_{2,\uparrow} \adop_{2,\downarrow}\aop_{1,\downarrow} \adop_{1,\downarrow}\adop_{2,\uparrow}}{\text{VAC}}\\
&=t
\end{aligned}
\]

This gives the Hamiltonian

\[
\braket{j}{H}{k} = 
\begin{pmatrix}
U & 0 & t & -t \\
0 & U & t & -t \\
t & t & 0 & 0 \\
-t & -t & 0 & 0
\end{pmatrix}
\]

Note that the off-diagonal block

\[
\begin{pmatrix}
t & -t \\
t & -t
\end{pmatrix}
\]

only connects the states
\(\frac{1}{\sqrt{2}}\left(\ket{1}+\ket{2}\right)\) and
\(\frac{1}{\sqrt{2}}\left(\ket{3}-\ket{4}\right)\), and we arrive at the
\(2\times 2\) matrix

\[
\begin{pmatrix}
U & 2t \\
2t & 0
\end{pmatrix}
\]

with eigenvalues \(U/2 \pm \sqrt{U^2/4+4t^2}\). Expanding for small
\(t/U\) gives \(U + \frac{4t^2}{U}\) and \(-4t^2/U\).

Note that the state \(\frac{1}{\sqrt{2}}\left(\ket{3}+\ket{4}\right)\)
is a spin-triplet, which is why its energy is unaffected, like the
states \(\adop_{1,\uparrow}\adop_{2,\uparrow} \ket{\text{VAC}}\) and
\(\adop_{1,\downarrow}\adop_{2,\downarrow} \ket{\text{VAC}}\) that we
discarded initially. The effect of finite \(t\) is therefore to lower
the energy of the singlet state by \(4t^2/U\) relative to the triplet
state.
\end{quote}

\end{tcolorbox}

\section{Effective Hamiltonian}\label{effective-hamiltonian}

As the number of sites increases, it becomes harder to say what happens
to the ground state multiplet. We now modify our strategy by splitting
the problem in two: we are going to find an \emph{effective Hamiltonian}
that acts only on the half filled Mott states and describes their
splitting when \(t/U\) is finite but small. Whether we can subsequently
solve that Hamiltonian we leave until later.

As I'm sure you realized when you thought about two sites, this is a
qualitatively different degenerate perturbation problem than the one we
solved when we added a single particle or hole to the bosonic Mott
states. The reason is that \(H_t\) has no matrix elements among the
degenerate states: when acting on one of them it always takes us into a
state with one site doubly occupied and the neighouring site empty. We
have to think about \emph{second order} degenerate perturbation theory
to find out what happens. To handle this we divide the Hamiltonian into
block form, according to whether its matrix elements act on the Mott
state or not.

\[
H = \begin{pmatrix}
H_{\text{Mott}} & V^{} \\
V^\dagger & H_\text{Not} \\
\end{pmatrix}
\]

Denoting by \(P_\text{Mott}\) the projection operator on to the
\(2^{N_\text{sites}}\) Mott states, and
\(P_\text{Not}\equiv 1-P_\text{Mott}\), we have

\[
\begin{aligned}
H_\text{Mott}= P_\text{Mott} H P_\text{Mott},\quad H_\text{Not}= P_\text{Not}H P_\text{Not}\\
V^{} = P_\text{Mott} H P_\text{Not},\qquad V^\dagger = P_\text{Not} H P_\text{Mott}.
\end{aligned}
\]

In the case of the Hubbard model, we have

\[
\begin{aligned}
H_\text{Mott}= P_\text{Mott} H_U P_\text{Mott},\quad H_\text{Not}= P_\text{Not}H P_\text{Not}\\
V^{} = P_\text{Mott} H_t P_\text{Not},\qquad V^\dagger = P_\text{Not} H_t P_\text{Mott}.
\end{aligned}
\]

Note that both \(H_U\) and \(H_t\) contribute to \(H_\text{Not}\)
because \(H_t\) can move particles and holes around in a non-Mott state.
We write the eigenvalue equation in block form

\[
\begin{pmatrix}
H_{\text{Mott}} & V^{} \\
V^\dagger & H_\text{Not} \\
\end{pmatrix}
\begin{pmatrix}
\ket{\Psi}\\
\ket{\Phi}
\end{pmatrix} = E
\begin{pmatrix}
\ket{\Psi}\\
\ket{\Phi}
\end{pmatrix}.
\]

We eliminate \(\ket{\Phi}\) to obtain

\[
\left[H_{\text{Mott}} -V^{}\left(H_\text{Not}-E\right)^{-1}V^\dagger\right]\ket{\Psi} = E\ket{\Psi}.
\]

So far we have made no approximation. While this looks like an
eigenvalue equation, we can't yet interpret the operator in the square
brackets as an effective Hamiltonian because it depends on the
eigenvalue \(E\). However, we now focus on energies much smaller than
the eigenvalues of \(H_\text{Not}\), which are \(O(U)\). In this way we
can neglect this energy dependence and arrive at the effective
Hamiltonian acting only on the Mott state

\[
H_\text{eff} = H_{\text{Mott}} -V^{} H^{-1}_\text{Not}V^\dagger.
\]

What form does \(H_\text{eff}\) take? \(H_{\text{Mott}}=0\), and
\(V^\dagger\) creates states with an adjacent hole and \textbf{doublon}
(doubly occupied site). \(H_\text{Not}\) acting on these states is just
\(U\), and \(V\) has to remove the hole and doublon. Thus,

\[
H_\text{eff} = -\frac{V^{}V^\dagger}{U} = -\frac{t^2}{U} \sum_{\substack{\langle j\,k\rangle\\s,s'}} \left[\adop_{j,s}\aop_{k,s} \adop_{k,s'}\aop_{j,s'}+j\leftrightarrow k\right].
\]

We can write this in a more familiar way by first reordering the
operators (not forgetting the anticommutation relations!)

\[
 \adop_{j,s}\aop_{k,s} \adop_{k,s'}\aop_{j,s'} = -\adop_{j,s}\aop_{j,s'}\adop_{k,s'}\aop_{k,s} + \delta_{s^{}s'}\adop_{j,s}\aop_{j,s'},\qquad j\neq k
\]

and then using the identity

\[
\delta_{ab}\delta_{cd} = \frac{1}{2}\left[\boldsymbol{\sigma}_{a d}\cdot \boldsymbol{\sigma}_{c b} + \delta_{ad}\delta_{cb}\right].
\]

Finally, in \(d\) dimensions (\(d=1\), chain; \(d=2\) square lattice;
\(d=3\) cubic lattice) we get

\[
H_\text{eff} = -\frac{dN_\text{sites}t^2}{U}+J\sum_{\langle j\,k\rangle} \mathbf{s}_j\cdot \mathbf{s}_k
\label{FilledHeff}
\]

with \(J=\frac{4t^2}{U}\) and

\[
\mathbf{s}_j=\frac{1}{2}\sum_{s,s'}\adop_{j,s}\boldsymbol{\sigma}_{s^{}s'}\aop_{j,s'}.
\]

The effective Hamiltonian is nothing but the spin-1/2 antiferromagnetic
Heisenberg model!

\begin{tcolorbox}[enhanced jigsaw, leftrule=.75mm, bottomrule=.15mm, colbacktitle=quarto-callout-tip-color!10!white, opacitybacktitle=0.6, colback=white, rightrule=.15mm, colframe=quarto-callout-tip-color-frame, toprule=.15mm, left=2mm, titlerule=0mm, title=\textcolor{quarto-callout-tip-color}{\faLightbulb}\hspace{0.5em}{Check}, arc=.35mm, breakable, bottomtitle=1mm, toptitle=1mm, opacityback=0, coltitle=black]

Note that there is something slightly sly about this derivation. We
assumed that the energy scale \(U\) was the largest scale in the
problem, in order to arrive at the effective Hamiltonian. However,
typical excited state energies of the Heisenberg Hamiltonian are
\(\frac{N_\text{sites}t^2}{U}\). Thus for
\(N_\text{sites}\gtrsim \left(\frac{t}{U}\right)^2\) there isn't
actually a separation between these two energies. Not a very useful
condition! Physically, it's enough to have a small density \(n\) of
doublons and holes, with overall energy \(\sim nU\), when \(t/U\) is
small.

\end{tcolorbox}

\section{Doping}\label{doping}

Antiferromagnetism and the Mott phenomenon are seen to go hand in hand
in fermion systems. This explains the common ocurrence of
antiferromagnetism in transition metal compounds, especially oxides. The
cuprate superconductors mentioned earlier are a famous example.

\begin{figure}[H]

{\centering \includegraphics[width=0.7\linewidth,height=\textheight,keepaspectratio]{../assets/Cuphasediag.png}

}

\caption{Schematic temperature vs.~doping diagram for the cuprate
materials
\href{https://en.wikipedia.org/wiki/High-temperature_superconductivity\#Cuprates}{{[}Source{]}}.}

\end{figure}%

At half filling, the cuprates are antiferromangetic Mott insulators.
Superconductivity emerges when the materials are doped by changing their
stoichiometry. This introduces electrons or holes into the CuO\(_2\)
planes that are modeled by the Hubbard Hamiltonian. Antiferromagnetic
order is believed to be destroyed by freely moving holes -- think how
the Néel ordering is disrupted -- and indeed superconductivity appears
where antiferromagnetism dies. The precise relationship between the two
phenomena is -- like much of the physics of the cuprates -- not clear.

An effective Hamiltonian that describes the doped Mott insulator is the
\href{https://en.wikipedia.org/wiki/T-J_model}{t-J model}

\begin{equation}\phantomsection\label{eq-filled-H-eff}{
H_\text{eff} = -t \sum_{\substack{\langle j,k\rangle\\ s=\uparrow,\downarrow}}  \left[\adop_{j,s}\aop_{k,s}+\adop_{k,s}\aop_{j,s}\right] + J\sum_{\langle j,k\rangle}\left[\mathbf{s}_j\cdot \mathbf{s}_k - \frac{N_j N_k}{4}\right].
}\end{equation}

The model is defined by supplementing the Hamiltonian with the
constraint that there are no doubly occupied sites. That is, we ignore
such states in the Hilbert space. This could be achieved by applying the
projector \(\prod_{j} (1-N_{j,\uparrow}N_{j,\downarrow})\). The term
involving \(N_j N_k\) keeps track of the need for both sites to be
occupied in the derivation of the effective coupling (c.f. it's just the
``constant'' term in Equation~\ref{eq-filled-H-eff}. The hopping term
means that holes or doubly occupied sites can move through the lattice,
with the Heisenberg exchange term only acting between sites with one
particle, since the operators \(\mathbf{s}_j\) vanish when they act on
an empty or doubly occupied site.


\backmatter


\end{document}
